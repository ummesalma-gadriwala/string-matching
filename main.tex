%++++++++++++++++++++++++++++++++++++++++
% Don't modify this section unless you know what you're doing!
\documentclass[letterpaper,12pt]{article}
\usepackage{tabularx} % extra features for tabular environment
\usepackage{amsmath}  % improve math presentation
\usepackage{graphicx} % takes care of graphic including machinery
\usepackage[margin=1in,letterpaper]{geometry} % decreases margins
\usepackage{cite} % takes care of citations
\usepackage[final]{hyperref} % adds hyper links inside the generated pdf file
\hypersetup{
	colorlinks=true,       % false: boxed links; true: colored links
	linkcolor=blue,        % color of internal links
	citecolor=blue,        % color of links to bibliography
	filecolor=magenta,     % color of file links
	urlcolor=blue         
}
%++++++++++++++++++++++++++++++++++++++++


\begin{document}

\title{CS 4TB3: Approximate Regular Expressions}
\author{Rumsha Siddiqui, Tasnim Noshin, and Umme Salma Gadriwala}
\date{\today}
\maketitle



\section{Description}


A regular expression is a special text string for describing a search pattern. Let \(n\) be the length of the text, \(m\) be the length of a regular expression \(R\) for the alphabet \(\sum\). Further, let \(d\) be the number of strings in \(R\), where a string is a sequence of characters connected by concatenation.
\\\\
The traditional technique to search for an exact regular expression in a text uses \(O(mn)\) worst case search time with a space requirement of \(O(m)\) states, by converting \(R\) into a non-deterministic finite automaton (NFA). An alternative algorithm converts the NFA into a deterministic finite automaton (DFA), and uses \(O(2^m)\) states and \(O(n)\) search time.
\\\\
Another interesting problem is approximate regular expression matching, that is searching for a given regular expression in a text allowing a limited number of errors \(k\), where \(k\) might be an insertion, a deletion or a substitution of a character by another. There exists a solution for this problem in time \(O(mn)\) and a solution for the case \(k = 0\) in time \(O(dn)\).
\\\\
To compare the performance of exact regular expression matching to approximate matching, we will implement the NFA-based algorithm and the Myers and Miller's algorithm.

\section{Resources}
\begin{itemize}
\item https://users.dcc.uchile.cl/~gnavarro/ps/wae99.pdf
\item https://www.sciencedirect.com/science/article/pii/S1570866712001116?via%3Dihub
\item https://www.data-essential.com/approximate-regular-expressions/
\end{itemize}


\section{Division of Work}


\subsection{Individual Tasks}
\begin{itemize}
\item  Tasnim Noshin: Implement exact algorithm
\item Umme Salma Gadriwala: Implement approximate algorithm
\item  Rumsha Siddique: Create test cases; Run the test cases; Look into literature to verify measuring techniquesr
\end{itemize}

   
\subsection{Group Tasks}
\begin{itemize}
\item  Hypothesis
\item Analyze results
\item  Final presentation and poster
\end{itemize}



\section{Weekly Schedule}


\begin{table}[ht]
\begin{center}

\label{tbl:bins} % spaces are big no-no withing labels
\begin{tabular}{|cc|} 
\hline
\multicolumn{1}{|c}{\textbf{Week}} & \multicolumn{1}{c|}{\textbf{Deliverable}} \\
\hline
week1 &  Implement algorithms and test cases \\
week2 &   Run test cases and tabulate results\\
week3 &   Analyze results; Write and submit report \\
week4 &   Prepare poster and presentation \\
\hline
\end{tabular}
\end{center}
\end{table}

Due date: April 10th






\end{document}
