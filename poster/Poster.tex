\documentclass[22pt]{beamer}
\usepackage{textcomp}
\usepackage[orientation=portrait, size=custom, width=91.44, height=91.44,scale=1.3]{beamerposter} % 36in*2.5 = 90cm
\usepackage[absolute,overlay]{textpos}
\usepackage{bookmark} %pdflatex says to use this to avoid errors...
\usepackage{graphicx} %for including images
\graphicspath{{figs/}} %location of images
\usepackage{wrapfig} %wrap text around the images
\usepackage{listingsutf8}    
\usepackage{amsmath}
\usepackage{gensymb}
\usepackage[export]{adjustbox}
\usepackage[skins,theorems]{tcolorbox}
\usepackage{tikz}
\newcommand*\circled[1]{\tikz[baseline=(char.base)]{
            \node[shape=circle,draw,inner sep=2pt] (char) {#1};}}
\usepackage{array}
\usepackage{booktabs,adjustbox}
\usepackage{subcaption} 


\usetheme{Berlin}
\definecolor{MacBlue}{rgb}{0.10196,0.22353,0.53725}
\definecolor{MacMaroon} {rgb}{0.47843, 0, 0.23137}
\definecolor{MacMaroon2} {rgb}{0.47451, 0, 0}
\definecolor{MacGray}{rgb}{0.50196,0.49804,0.51765}
\definecolor{MacMaroon3}{rgb}{00.47,0.2,0.31}
\definecolor{MacGold}{rgb}{1, 0.75,0.35}
\usecolortheme[named=MacBlue]{structure}
\setbeamertemplate{caption}[numbered]
\setbeamertemplate{navigation symbols}{}

\title{Approximate Regular Expressions: A Comparison of Exact and Approximate Matching Algorithms}
\subtitle{}  
  \author[Gadriwala, Noshin \& Siddiqui]{Umme Salma Gadriwala, Tasnim Noshin, Rumsha Siddiqui \vspace{0.3cm} \newline \small \{gadriway, noshint, siddiqur\}@mcmaster.ca}
  \institute[McMaster University]{Department of Computing and Software, McMaster University

1280 Main St. W, Hamilton, Ontario, Canada L8S 4L8}
  \date{April 2019}

\begin{document}
\begin{frame}[fragile]

\begin{textblock}{2}(0.7,1)
\includegraphics[height=8.5cm]{englogo.png}
\end{textblock}

\begin{textblock}{8}(4,1)
\titlepage
\end{textblock}

\begin{textblock}{2}(12.7,0.75)
\includegraphics[height=12cm]{cslogo.png}
%\centering
%\textbf{Tweet us} \\
%your thoughts: \\
%{[worth pursuing?] [achievable?]}
%{\color{MacBlue} @TeamDeepCheck \#MacCSCapstone \#MacEng}
\end{textblock}

\begin{textblock}{7.25}(0.5,3.1)

\begin{block}{Motivation}
Applications of finding exact and approximate string matches range far and wide. In browsing, the find function is often employed to locate exact text snippets on a screen. In computer security, approximate matching is necessary to find computer viruses and spam signatures that resemble certain patterns. In bioinformatics, approximate matching is vital for comparing and contrasting DNA and protein sequences \cite{Approx1}. The length of sequences involved in these applications reach billions. Thus, efficient string matching algorithms are needed to preserve time and cost of resources.
\end{block}

\begin{block}{Problem}
 Different applications have different constraints on how strict of a match must be found. Some applications require both exact and approximate matching features. This project explores whether or not an implementation of exact matching can be replaced with one of approximate matching, where the error value for an exact match request would be set to zero. We aim to determine how an approximate matching algorithm fares in efficiency when compared to an exact matching algorithm.  
 %For ex: browser
 	% exact matching: ctrl-f to find text snippet
 	% approx. matching: allow for typos

\end{block}

\begin{block}{Solution}
A Python implementation of Thompson's exact matching algorithm is compared with that of Myers and Miller's approximate matching algorithm. Matches between string and regular expressions of various legnths are used for testing. The testing methodology compares the time each algorithm takes for the same samples. 
\end{block}


\begin{block}{Background Study}
\textbf{Exact Matching: Thompson's Algorithm}:


%\begin{figure}
%\includegraphics[scale=1.5]{ThompsonsNFA.PNG}
%\end{figure}

\vspace{5mm} %5mm vertical space



\textbf{Approximate Matching: Myers and Miller's Algorithm}:


\begin{figure}
\includegraphics[scale=2]{MillerAndMyersAlgorithm.PNG} 
%\includegraphics[scale=1]{ThompsonsNFA.PNG}
\hspace*{.2in}
\raisebox{0.7\height}{\includegraphics[scale=1.2]{ThompsonsNFAcropped.PNG}}
\end{figure}

\end{block}
\end{textblock}


\begin{textblock}{7.25}(8.25,3.1)
\begin{block}{Methodology}

explain what doing

graph

\end{block}


\begin{block}{Conclusions \& Future Work}
final result

future:other algos
\end{block}


\begin{block}{Acknowledgements}
We thank Dr. Sekerinski for inspiring this project, as well as his continued guidance in our learning and endeavours. We also thank Spencer Park and Erin Varey for their invaluable feedback and support. 
\end{block}

%--------------------------------------------
%REFERENCES
\begin{block}{References}
\setbeamertemplate{bibliography item}{\insertbiblabel}
\bibliographystyle{ieeetr}
{\scriptsize
\bibliography{bib}}
\end{block}

\begin{comment}
\begin{block}{References}
\setbeamertemplate{bibliography item}{\insertbiblabel}
\bibliographystyle{ieeetr}
{\scriptsize
\bibliography{bib}}
\end{block}
\vspace{-1.8mm}
\end{comment}
\end{textblock}

%
%\begin{textblock}{7.5}(0.5,14.6)
%\centering
%\textbf{Tweet us} \\
%your thoughts on DeepCheck and where you think the future of neural networks is headed \\
%{\color{MacBlue} @TeamDeepCheck \#MacCSCapstone \#MacEng}
%
%\end{textblock}

\end{frame}
\end{document}
